\documentclass{article}
\usepackage{url,amsfonts, amsmath, amssymb, amsthm, graphicx, braket, fancyhdr, physics, mathtools}
\graphicspath{ {./QM_Images/} }
% Page layout
\setlength{\textheight}{8.75in}
\setlength{\columnsep}{2.0pc}
\setlength{\textwidth}{6.5in}
\setlength{\topmargin}{0in}
\setlength{\headheight}{0.0in}
\setlength{\headsep}{0.0in}
\setlength{\oddsidemargin}{0in}
\setlength{\evensidemargin}{0in}
\setlength{\parindent}{1pc}

% Commands
\newcommand{\RNum}[1]{\uppercase\expandafter{\romannumeral #1\relax}}

\usepackage[a4paper,margin=2.5cm,portrait, headsep=24pt, headheight=2cm]{geometry}
\usepackage{enumitem}

% Header
\pagestyle{fancy}
\fancyhf{}
\rhead{Page \thepage}
\chead{Modern Physics DE}
\lhead{James Natoli}
\rfoot{}

\begin{document}
% suppress header on first page
\thispagestyle{empty}
%%% Title Stuff
\begin{center}\Large \textbf{Modern Physics Departmental Exam} \\
\normalsize James Natoli \\  January 2021
\end{center}

%%%%%%%%%%%%%%%%% Problem  1 %%%%%%%%%%%%%%%%%
\section*{Problem 1} 
% Problem Statement
The muon is an unstable particle that spontaneously decays into an electron and two neutrinos. If the number of muons at $t=0$ is $N_0$, the number at time $t$ is given by $N=N_0e^\frac{t}{\tau}$, where $\tau$ is the mean lifetime, equal to 2.2 $\mu$s. Suppose the muons move at a speed of $0.95c$ and there are $5.0 \cdot 10^4$ muons at $t=0$
\begin{enumerate}[label=\alph*)]
	\item % PART (A)
	What is the observed lifetime of the muons?
	\item % PART(B)
	How many muons remain after traveling a distance of 3.0 km?
\end{enumerate}
%% Solution Here
\section*{\textit{Solution}} 
\begin{enumerate}[label=\alph*)]
	\item % PART (A)
	This is a time dilation problem, just need to know the formula
	$$ t' = \gamma t $$
	where $\gamma = \frac{1}{\sqrt{1 - (v/c)^2}}$, $t'$ is the dilated time, and $t$ is the regular time
	$$ t' = 2.2 \times \frac{1}{\sqrt{1 - (0.95)^2}}$$
	$$ t' = 7.0456 \text{ }\mu s$$
	To help check your answer, it should always be a longer time than the regular time
	\item % PART (B)
	The muons will see a shorter distance, due to length contraction
	$$ L' = \frac{L}{\gamma} $$ 
	where $\gamma$ is same as before, $L'$ is the contracted length, and $L$ is the regular length
	$$ L' = 3\times 10^3 \cdot \sqrt{1 - 0.95^2} $$
	$$ L' = 936.75 \text{ m} $$
	We need a time for the formula given in the prompt, $N=N_0e^\frac{t}{\tau}$, so we need to find the time it takes for the muons to travel $L'$ at $0.95c$
	$$ \frac{L'}{v} = \text{ time } = \frac{936.75}{0.95c} = 3.28 \cdot 10^{-6} \text{ s}$$
	$$ N(t) = N_0e^\frac{t}{\tau} = 5\cdot 10^{4} \cdot e^\frac{3.28 \cdot 10^{-6}}{2.2\cdot10^{-6}} $$
	$$ N(3.28 \cdot 10^{-6}) = 1.1223 \cdot 10^4 \text{ muons}$$

\end{enumerate}

%%%%%%%%%%%%%%%%% Problem 2 %%%%%%%%%%%%%%%%%
\section*{Problem 2} 
% Problem Statement
X-ray photons of wavelength 0.02580 nm are incident on a target and the Compton-scattered photons are observed at $90^{\circ}$
\begin{enumerate}[label=\alph*)]
	\item % PART (A)
	What is the wavelength of the scattered photons?
	\item % PART (B)
	What is the momentum of the incident photons? Of the scattered photons?
	\item % PART (C)
	What is the kinetic energy of the scattered electrons?
	\item % PART (D)
	What is the momentum (magnitude and direction) of the scattered electrons?
\end{enumerate}
%% Solution Here
\section*{\textit{Solution}} 
\begin{enumerate}[label=\alph*)]
	\item % PART (A)
\end{enumerate}

%%%%%%%%%%%%%%%%% Problem  %%%%%%%%%%%%%%%%%
\section*{Problem } 
% Problem Statement
\begin{enumerate}[label=\alph*)]
	\item % PART (A)
\end{enumerate}
%% Solution Here
\section*{\textit{Solution}} 
\begin{enumerate}[label=\alph*)]
	\item % PART (A)
\end{enumerate}


\end{document}










